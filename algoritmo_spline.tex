\begin{verbatim}
#include <fstream>
#include "Parametros.h"
#include "Spline.h"

using namespace std;

int main(){
    Spline spl(nPontos); //declaracao e passagem de parametro para construtor da Classe Spline.
    spl.leituraPontos(); //Le do arquivo pontos.txt, os pontos que serão interpolados.
    spl.exibePontos(); //Exibe na tela os pontos
    spl.passo1(); // gera os h's, pela equacao hi = xi+1 - xi
  /*  spl.passo2();  // passo de 2 a 6 é a resolução do sistema linear.
    spl.passo3e4();
    spl.passo5e6();*/
    spl.preparandoThomas(); // prepara o sistema e utiliza o algoritmo de thomas pra resolver-lo.
    spl.saidaSpline(); // saida na tela e em arquivo .txt, dos coeficientes das do spline Cubico.
    spl.splineCubicoGNU(); // gera um arquivo .gnu para desenhar o gráfico da solucao. 
     
    system("PAUSE");
    return EXIT_SUCCESS;
}

Spline.h

// Classe Spline
#include "Parametros.h"

#ifndef SPLINE

#define SPLINE

class Spline{
    private: //Atributos Privados
      int n;
      double* x;
      double* y;
      double* h;
      double* a;
      double* b;
      double* c;
      double* d;    
      double* alfa;   
      double* beta;   
      double* gama;  
      double* delta;   
    public: //Metodos Publicos
    
      Spline(int ns); // Construtor da Classe Spline
      void leituraPontos(void); // Le os pontos do arquivo "ponto.txt"      
      void exibePontos(void); // Exibe na tela esses pontos
      
      //* Gera os h's, espaços entre os x's
      void passo1(void);
      
      //* Metodos usados para resolver o Sistema Linear
      void passo2(void);
      void passo3e4(void);
      void passo5e6(void);
      //***************************
      
      //* Metodos usados para resolver o Sistema Linear, usando algorimto de Thomas
      void algoritmoThomas(double *A,double *B, double *C, double *D, double *X, long int N);
      void preparandoThomas(void);
      //***************************
      
      //* Exibe os coeficientes dos polinomios
      void saidaSpline(void);
      //* Gera um arquivo .gnu para ser plotado
      void splineCubicoGNU(void);

};

#endif


Spline.cpp

#include "Spline.h"
#include <fstream>
#include <iostream>

using std::cout;
using std::cin;
using std::endl;

using namespace std;

Spline::Spline(int ns){ // Construtor
   n = ns - 1;  // sao ns pontos(qntdade de pontos), mas a referencia sera de ns-1 para os metodos
   x = new double[n+1];  // vetor x, armazena os valores do eixo x   
   y = new double[n+1];  // vetor y, armazena os valores do eixo y, os f(x).
   h = new double[n+1];  // vetor h, que armazena os espaços entres os x's.
   a = new double[n+1];  // equivale aos f(x), aj = f(xj), sao termos independentes dos polinomios gerados
   b = new double[n+1];  // sao os valores que multiplicam o termo linear, bj*(x-xj)
   c = new double[n+1];  // sao os valores que multiplicam o termo quadratico, cj*(x-xj)^2
   d = new double[n+1];  // sao os valores que multiplicam o termo cubico, dj*(x-xj)^3
   //Nota: S(x) = Sj(x) = aj + bj(x - xj) + cj(x - xj)^2 + dj(x - xj)^3
   
   // Sao variaveis auxiliares usadas no algoritmo 1, para resolver o sistema linear
   alfa = new double[n+1];  
   beta = new double[n+1];
   gama = new double[n+1];
   delta = new double[n+1];
}

void Spline::leituraPontos(void){ 
     // classe ifstream, de leitura
     ifstream arq("pontos.txt");  //Leitura dos pontos do problema, no arquivo pontos.txt
     for (int i=0; i <= n; i++){
         if (! arq.eof()){
            arq >> x[i];
            arq >> y[i];
            a[i] = y[i]; // ai = f(xi)
         }
     }
}

void Spline::exibePontos(void){   // Exibe na tela os pontos q foram lidos.
     cout << "j        xj      f(xj) \n";
     cout << "----------------------\n";
     for (int i=0; i <= n; i++){
         printf("%d\t%0.2f\t%0.2f\n",i,x[i],y[i]);
     }
     cout << "\nPrecione Enter pra executar o metodo Spline Cubica.";
     getchar();
     cout << "\n\n";
}

//Nota: S(x) = Sj(x) = aj + bj(x - xj) + cj(x - xj)^2 + dj(x - xj)^3
void Spline::passo1(void){ // Calcura os h's, distancia entres os x's
     for (int i=0; i < n; i++){
         h[i] = x[i+1] - x[i];
     }
}

/*** ALGORITMO 1 - Resolucao do Sistema Linear ***/
// Consiste nos Passos de 2 a 6
void Spline::passo2(void){ 
     for (int i=1; i < n; i++){
         alfa[i] = (3.0/h[i])*(a[i+1] - a[i]) - (3.0/h[i-1])*(a[i] - a[i-1]);
     }
}

void Spline::passo3e4(void){ 
     beta[0] = 1.0;
     gama[0] = 0.0;
     delta[0] = 0.0;
     for (int i=1; i < n; i++){
         beta[i] = 2.0*(x[i+1] - x[i-1]) - h[i-1]*gama[i-1];
         gama[i] = h[i]/beta[i];
         delta[i] = (alfa[i] - h[i-1]*delta[i-1])/beta[i];
     }
}

void Spline::passo5e6(void){ 
     beta[n] = 1.0;
     gama[n] = 0.0;
     delta[n] = 0.0;
     for (int j=n-1; j >=0 ; j--){
         c[j] = delta[j] - gama[j]*c[j+1];
         b[j] = (a[j+1] - a[j])/h[j] - h[j]*(c[j+1] + 2.0*c[j])/3.0;
         d[j] = (c[j+1] - c[j])/(3*h[j]);
     }
}

\end{verbatim}